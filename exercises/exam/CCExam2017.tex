\documentclass [11pt, a4wide, twoside]{article}

\usepackage{times}
%\usepackage{epsfig}
\usepackage{ifthen}
%\usepackage{xspace}
\usepackage{fancyhdr}
\usepackage{graphicx}
\usepackage{setspace}
\usepackage[colorlinks,urlcolor=blue]{hyperref}

\usepackage{verbatim} %For comments 

% solution switch
\newboolean{showsolution}
%\setboolean{showsolution}{true} 
\setboolean{showsolution}{false} 

%layout
\topmargin      -5.0mm
\oddsidemargin  6.0mm
\evensidemargin -6.0mm
\textheight     215.5mm
\textwidth      160.0mm
\parindent      1.0em
\headsep        10.3mm
\headheight     12pt
\lineskip       1pt
\normallineskip 1pt

%header
\lhead{Compiler Construction \\ 2017}

\rhead{\small Prof. O. Nierstrasz, Mohammad Ghafari\\
        Yuriy Tymchuk, Manuel Leuenberger}

\lfoot{page \thepage}
\rfoot{\today}
\cfoot{}

\renewcommand{\headrulewidth}{0.1pt}
\renewcommand{\footrulewidth}{0.1pt}

\def\xexercise{\fontsize{12}{10}\fontseries{bx}\selectfont}
\def\xnormal{\fontseries{m}\fontshape{n}\selectfont}

\newcounter{exnum}
\newcommand{\exercise}[1]{%
	{\addtocounter{exnum}{1}\vskip 0.8cm{\xexercise \noindent Exercise \arabic{exnum} (#1)} \xnormal} \vskip 0.3cm}
\newcommand{\aufgabe}[1]{
	{\addtocounter{exnum}{1}\vskip 0.8cm{\xexercise \noindent Aufgabe \arabic{exnum} (#1)} \xnormal} \vskip 0.3cm}

%enumeration
\newenvironment{myitemize}{%
     \begin{itemize}
     \setlength{\itemsep}{0cm}}
     {\end{itemize}}

\newenvironment{myenumerate}{%
	\renewcommand{\labelenumi}{\alph{enumi})}
	\renewcommand{\labelenumii}{\arabic{enumii})}
     \begin{enumerate} \setlength{\itemsep}{0cm}}
     {\end{enumerate}}


%solution
\ifthenelse{\boolean{showsolution}}
  {  \newcommand{\solution}[1]{
         \noindent\underline{\textbf{Answer:}}\\[2mm]
         \textsl{#1}
         \vspace{10pt}
         \normalsize
	   }
  }
  {  
    \newcommand{\solution}[1]{} 
  }

\ifthenelse{\boolean{showsolution}}
{
  \newenvironment{solutionBlock}{
         \noindent\underline{\textbf{Answer:}}
  }{}
}{
  \let\solutionBlock\comment
}


\pagestyle{fancy}

\usepackage[usenames,dvipsnames]{xcolor}
\definecolor{light-gray}{gray}{0.95}
\newcommand\code[1]{\colorbox{light-gray}{\texttt{#1}}} % code
\newcommand\ccun{$\cup$} % code
\newcommand\ccin{$\cap$} % code

\graphicspath{{./figures/}}

\begin{document}

% title
\section*{\ifthenelse{\boolean{showsolution}}{Solution }{}COMPILER CONSTRUCTION - EXAM}

% - - - - - - - - - - - - - - - - - - - - - - - - - - - - - - - - - - - - - - - - - - - - - - - - - - - - - - - - - - - - - - - - - - -

First name: \rule{100pt}{0.5pt}\\[1mm]
Last name: \rule{100pt}{0.5pt}\\[1mm]
Matrikel: \rule{100pt}{0.5pt}\\[1mm]
\rule{\textwidth}{1pt}\\[1mm]
Date: Friday, 02.06.2017\\
Allowed material: this paper and a pen\\
Number of exercises: 3\\
Total points: 30\\[1mm]
\noindent\textbf{Important:\\
You have 90 minutes to solve the exam.\\
Some exercises are split over more than one page, so carefully read all exercises before proceeding. 
If you don't know something don't waste time, go ahead and if you have time go back and try to solve unanswered questions afterwards.
You must answer each question briefly and to the point. 
%Remember to put your first name and last name on any other sheet that you use. Answers on sheets without your id will not be corrected.
}\\
\rule{\textwidth}{1pt}

% 
\newpage
\exercise{10 Points}
\noindent
%
Answer the following questions, any answer that exceeds 3 lines will not be corrected. Seriously!

\begin{myenumerate}

\item Why is it important that a compiler supports separate compilation? \textbf{(2 Points)}
\solution{Because you don't have to recompile the whole system on a slight change.}\vspace{3cm}

\item What is the difference between context-free and context-sensitive grammars? \textbf{(2 Points)}
\solution{Context sensitive can have both terminals and non-terminals on the left-hand side of the transformations, thus meaning of the can change depending on the context and we cannot parse it with a pushdown automaton}\vspace{3cm}

\item Which kind of parsers have a problem with left-recursive grammars? What are the effects of the problem? \textbf{(2 Points)}
\solution{}\vspace{3cm}

\item What is SSA and which optimizations rely on it?\textbf{(2 Points)}
\solution{Single Static Assignment, each variable is only assigned once. Enables constant propagation, dead code elimination, efficient register management}\vspace{3cm}

\item Why does a generational garbage collector distinguish between old and new objects? \textbf{(2 Points)}
\solution{Objects either live forever or die shortly after creation. Check the old generation less than the new generation to minimize objects to check.}\vspace{3cm}

\newpage


\end{myenumerate}

\exercise{7 Points}
\noindent
%
Consider a language over an $0$ and $1$ alphabet that always starts with $01$ and ends with $10$, and has no consecutive $0$s. For example:
\begin{verbatim}
0110, 01010, 0111111110, 010101010, 01111010110110, ...
\end{verbatim}
But not:
\begin{verbatim}
010
\end{verbatim}

\begin{myenumerate}

\item Write a regular expression describing the language. \textbf{(2 Point)}\vspace{3cm}

\begin{solutionBlock}
\begin{verbatim}
01(1|01)*(0)?10 -> 01(1|01)+0
\end{verbatim}
\end{solutionBlock}

\item Draw a corresponding non-deterministic finite state automaton (NFA). \textbf{(2 Points)}\vspace{5cm}
\solution{http://hackingoff.com/compilers/regular-expression-to-nfa-dfa}

\newpage
\item Derive a corresponding deterministic finite state automaton (DFA) from the NFA in the previous step. Show the intermediate steps in the construction. \textbf{(3 Points)}\vspace{5cm}
\solution{http://hackingoff.com/compilers/regular-expression-to-nfa-dfa}

\end{myenumerate} 

% 
\newpage
\exercise{13 Points}
\noindent

Consider simplified boolean algebra with operations $or$ ($\lor$), $and$ ($\land$), $parentheses$, and literals $T$ and $F$.
$\land$ has a higher precedence than $\lor$.
See the examples:

\begin{itemize}
	\item $T$
	\item $T \lor F \land T$
	\item $(T)$
	\item $(T \lor F \lor F) \land (F)$
	\item $(F \land F) \land (F \lor T) \lor T \land F$
\end{itemize}

\begin{myenumerate}
\item Write a context free grammar \textbf{(2 Points)}\vspace{6cm}

\begin{solutionBlock}
\begin{verbatim}
expr     := '(' expr ')' |
            expr '∧' expr |
            expr '∨' expr |
            bit
bit      := 'T' |
            'F'
\end{verbatim}
\end{solutionBlock}


\newpage
\item Write a context free grammar with precedence and without any left recursion: \textbf{(4 Points)}\vspace{6cm}

\begin{comment}
\begin{verbatim}
With precedence:
expr     := expr '∨' exprAnd |
            exprAnd
exprAnd  := exprAnd '∧' exprPar |
            exprPar
exprPar  := '(' expr ')' |
            bit
bit      := 'T' |
            'F'

remove left recursion:
expr     := exprAnd expr'
expr'    := '∨' exprAnd expr' |
            ε
exprAnd  := exprPar exprAnd'
exprAnd' := '∧' exprPar exprAnd' |
            ε
exprPar  := '(' expr ')' |
            bit
bit      := 'T' |
            'F'
\end{verbatim}
\end{comment}


\newpage
\item Draw a concrete syntax tree (CST) for the following example with respect to the grammar of the previous point. 
Example: $F \land (T \lor F) \lor T$. \textbf{(3 Points)}\vspace{8cm}

\begin{comment}
see 3-trees.png
\end{comment}

\newpage


\item Explain how you would convert your CST into an AST. Provide the AST of the CST that you constructed for the previous step. \textbf{(2 Points)} \vspace{7cm}

\begin{comment}
\begin{verbatim}
rewrite CST to AST with and or nodes and true false children by pushing consumed operator in expr' and axprAnd' to parent node (see 3-trees.png)
\end{verbatim}
\end{comment}


\item Describe an interpreter that evaluates the AST into $true$ or $false$. \textbf{(2 Points)}
\\Here are some examples what are the expected results:
\begin{itemize}
	\item $T$
		= $true$
	\item $T \lor F \land T$
		= $true$
	\item $(T)$
		= $true$
	\item $(T \lor F \lor F) \land (F)$
		= $false$
	\item $(F \land F) \land (F \lor T) \lor T \land F$
		= $false$
\end{itemize}
\vspace{3.8cm}

\end{myenumerate}

\begin{comment}
\begin{verbatim}
apply visitor

bit -> true, false
and -> left.evaluate() && right.evaluate()
or -> left.evaluate() || right.evaluate()
\end{verbatim}
\end{comment}

% - - - - - - - - - - - - - - - - - - - - - - - - - - - - - - - - - - - - - - - - - - - - - - - - - - - - - - - - - - - - - - - - - - -
\newpage
\subsection*{Points}

\begin{minipage}[t]{120pt}
\textbf{Exercise 1}
\vspace{5pt}\\
\begin{tabular}{|c|c|c|c|}
\hline
Task & Points & Score \\\hline
1 & 2 & \\\hline
2 & 2 & \\\hline
3 & 2 & \\\hline
4 & 2 & \\\hline
5 & 2 & \\\hline
\textbf{Total} & \textbf{10} &\\\hline\hline
\end{tabular}
\end{minipage}
\begin{minipage}[t]{120pt}


% \textbf{Exercise 2}
% \vspace{5pt}\\
% \begin{tabular}{|c|c|c|c|}
% \hline
% Task & Points & Score \\\hline
% 1 & 3 & \\\hline
% \textbf{Total} & \textbf{3} &\\\hline\hline
% \end{tabular}
% \end{minipage}
% \begin{minipage}[t]{120pt}


\textbf{Exercise 2}
\vspace{5pt}\\
\begin{tabular}{|c|c|c|c|}
\hline
Task & Points & Score \\\hline
1 & 2 & \\\hline
2 & 2 & \\\hline
3 & 3 & \\\hline
\textbf{Total} & \textbf{7} &\\\hline\hline
\end{tabular}
\end{minipage}
\begin{minipage}[t]{120pt}


\textbf{Exercise 3}
\vspace{5pt}\\
\begin{tabular}{|c|c|c|c|}
\hline
Task & Points & Score \\\hline
1 & 2 & \\\hline
2 & 4 & \\\hline
3 & 3 & \\\hline
4 & 2 & \\\hline
5 & 2 & \\\hline
\textbf{Total} & \textbf{13} &\\\hline\hline
\end{tabular}
\end{minipage}

\noindent
%\begin{minipage}[t]{120pt}
%
%\vspace{1cm}
%
%\textbf{Exercise 5}
%\vspace{5pt}\\
%\begin{tabular}{|c|c|c|c|}
%\hline
%Task & Points & Score \\\hline
%1 & 2 & \\\hline
%2 & 2 & \\\hline
%\textbf{Total} & \textbf{4} &\\\hline\hline
%\end{tabular}
%\end{minipage}
%\begin{minipage}[t]{120pt}
%\vspace{1cm}
%
%\textbf{Exercise 6}
%\vspace{5pt}\\
%\begin{tabular}{|c|c|c|c|}
%\hline
%Task & Points & Score \\\hline
%1 & 2 & \\\hline
%2 & 2 & \\\hline
%3 & 1 & \\\hline
%\textbf{Total} & \textbf{5} &\\\hline\hline
%\end{tabular}
%\end{minipage}

\end{document}
