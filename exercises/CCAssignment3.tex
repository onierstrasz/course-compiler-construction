\documentclass[a4wide]{scrartcl}


% Page setup
%%%%%%%%%%%%
\usepackage{geometry}
\geometry{
  bottom=2.5cm,
  top=2.5cm,
  left=2.5cm,
  right=2.5cm,
  includeheadfoot
}

%% Fonts (requires lualatex or xelatex!)
%%%%%%%%%%%%%%%%%%%%%%%%%%%%%%%%%%%%%%%%
%\usepackage{fontspec}
%\setmonofont{DejaVu Sans Mono}

% Header and footer
%%%%%%%%%%%%%%%%%%%
\usepackage{fancyhdr}
\lhead{\small Compiler Construction \\ Spring Semester 2019}
\rhead{\small Prof. Dr. O. Nierstrasz, Dr. Mohammad Ghafari\\
        Manuel Leuenberger, Claudio Corrodi}
\cfoot{\thepage}
\usepackage{tabularx}
\usepackage[hidelinks]{hyperref}
\renewcommand{\headrulewidth}{0.1pt}
\renewcommand{\footrulewidth}{0.1pt}
\pagestyle{fancy}


%% Comments, notes, etc.
%%%%%%%%%%%%%%%%%%%%%%%%
\usepackage{ifthen}
\newboolean{showcomments}
\setboolean{showcomments}{true}
%\setboolean{showcomments}{false}

\usepackage{xcolor}
\usepackage{amssymb}

\ifthenelse{\boolean{showcomments}}{
  % then 
  \newcommand{\bnote}[2]{
    \fbox{\bfseries\sffamily\scriptsize#1}
    {\sffamily\small$\blacktriangleright$\textit{#2}$\blacktriangleleft$}
  }
}{
  % else
  \newcommand{\bnote}[2]{}{}
}
  
\newcommand{\triangles}[1]{{\sffamily\small$\blacktriangleright$\textit{#1}$\blacktriangleleft$}}
\newcommand{\nbc}[3]{
  \ifthenelse{\boolean{showcomments}}{
    % then    
    {\colorbox{#3}{\bfseries\sffamily\scriptsize\textcolor{white}{#1}}}%
    {\textcolor{#3}{\sffamily\small$\blacktriangleright$\textit{#2}$\blacktriangleleft$}}
  }{
    % else
  }
}

% Notes
\newcommand\cc[1]{\nbc{CC}{#1}{green!50!black}}
\newcommand\ml[1]{\nbc{ML}{#1}{cyan!80!black}}
\newcommand\todo[1]{\nbc{TODO}{#1}{red}}
\newcommand\verify[1]{\nbc{TO VERIFY}{#1}{red!50!black}}

% Code listings
%%%%%%%%%%%%%%%
\usepackage{listings}
\lstset{
  basicstyle=\ttfamily
}


% Sane quotes
%%%%%%%%%%%%%
\usepackage{csquotes}


% Shorter titles
%%%%%%%%%%%%%%%%
\makeatletter
\def\@maketitle{%
  \noindent
  {\centering
  {\LARGE \sffamily \textbf{\@title}}\\[0.75em]
  {\Large \sffamily \textit{\@subtitle}}\par
  }
}
\makeatother

% Keep the header when inserting the title
\newcommand\makefancytitle{%
  \maketitle\vspace{0em plus 2em minus 2em}%
  \thispagestyle{fancy}%
}

% Include graphics in various formats
\usepackage{graphicx}


% Default title; use subtitle for assignment specific titles
\title{Compiler Construction 2019}
\subtitle{Assignment 3: Parsing}
\begin{document}

\makefancytitle

\section{Left recursion}

\cc{Same as in 2017.}

Consider the following grammar.
\begin{lstlisting}
<sentence> ::= <words>
<words> ::= <words><word> | <word>
\end{lstlisting}

Do the following tasks.
\begin{enumerate}
\item Remove the left recursion.
\item Why are left recursions bad? For what type of parsers? Keep the answer short and precise.
\end{enumerate}


\section{Extending the grammar}

\cc{Same as in 2017.}

Extend the grammar from Exercise 1 so it can support questions (sentences terminated with a question mark), exclamations (sentences terminated with a exclamation mark), complex sentences (parts are divided by a comma), and the notion that the first word of a sentence must begin with a capital letter. Also, any other word in the sentence can begin with a capital letter. Assume that \texttt{<capitalWord>} is a word with a capital first letter.

Extra task (not graded): Write regular expressions for \texttt{<capitalWord>} and \texttt{<word>}.


\section{Parsing mathematical expressions in reverse Polish notation}

Write a grammar for parsing mathematical expressions in Reverse Polish Notation\footnote{\url{https://en.wikipedia.org/wiki/Reverse_Polish_notation}} (RPN).
Your grammar should be able to parse expresions that include digits the binary operators * (multiplication), - (subtraction), + (addition), / (division), as well as the unary operators N (negation) and ! (factorial).

Valid expressions:
\begin{itemize}
\item 3 4 +
\item 10 2 + ! 3 4 + * 5 6 * /
\item 2 ! 3 ! +
\item 12 N
\item 12 13 -
\item 1 2 N - ! N
\end{itemize}

Invalid expressions:
\begin{itemize}
\item 3 + 4
\item + 1 2
\item 1 +
\item 5 -
\end{itemize}

Extra task (not graded): Ensure that your grammar does not contain any left recursion.

\end{document}
%%% Local Variables:
%%% mode: latex
%%% TeX-master: t
%%% End:
