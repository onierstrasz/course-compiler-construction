\documentclass[a4wide]{scrartcl}


% Page setup
%%%%%%%%%%%%
\usepackage{geometry}
\geometry{
  bottom=2.5cm,
  top=2.5cm,
  left=2.5cm,
  right=2.5cm,
  includeheadfoot
}

%% Fonts (requires lualatex or xelatex!)
%%%%%%%%%%%%%%%%%%%%%%%%%%%%%%%%%%%%%%%%
%\usepackage{fontspec}
%\setmonofont{DejaVu Sans Mono}

% Header and footer
%%%%%%%%%%%%%%%%%%%
\usepackage{fancyhdr}
\lhead{\small Compiler Construction \\ Spring Semester 2019}
\rhead{\small Prof. Dr. O. Nierstrasz, Dr. Mohammad Ghafari\\
        Manuel Leuenberger, Claudio Corrodi}
\cfoot{\thepage}
\usepackage{tabularx}
\usepackage[hidelinks]{hyperref}
\renewcommand{\headrulewidth}{0.1pt}
\renewcommand{\footrulewidth}{0.1pt}
\pagestyle{fancy}


%% Comments, notes, etc.
%%%%%%%%%%%%%%%%%%%%%%%%
\usepackage{ifthen}
\newboolean{showcomments}
\setboolean{showcomments}{true}
%\setboolean{showcomments}{false}

\usepackage{xcolor}
\usepackage{amssymb}

\ifthenelse{\boolean{showcomments}}{
  % then 
  \newcommand{\bnote}[2]{
    \fbox{\bfseries\sffamily\scriptsize#1}
    {\sffamily\small$\blacktriangleright$\textit{#2}$\blacktriangleleft$}
  }
}{
  % else
  \newcommand{\bnote}[2]{}{}
}
  
\newcommand{\triangles}[1]{{\sffamily\small$\blacktriangleright$\textit{#1}$\blacktriangleleft$}}
\newcommand{\nbc}[3]{
  \ifthenelse{\boolean{showcomments}}{
    % then    
    {\colorbox{#3}{\bfseries\sffamily\scriptsize\textcolor{white}{#1}}}%
    {\textcolor{#3}{\sffamily\small$\blacktriangleright$\textit{#2}$\blacktriangleleft$}}
  }{
    % else
  }
}

% Notes
\newcommand\cc[1]{\nbc{CC}{#1}{green!50!black}}
\newcommand\ml[1]{\nbc{ML}{#1}{cyan!80!black}}
\newcommand\todo[1]{\nbc{TODO}{#1}{red}}
\newcommand\verify[1]{\nbc{TO VERIFY}{#1}{red!50!black}}

% Code listings
%%%%%%%%%%%%%%%
\usepackage{listings}
\lstset{
  basicstyle=\ttfamily
}


% Sane quotes
%%%%%%%%%%%%%
\usepackage{csquotes}


% Shorter titles
%%%%%%%%%%%%%%%%
\makeatletter
\def\@maketitle{%
  \noindent
  {\centering
  {\LARGE \sffamily \textbf{\@title}}\\[0.75em]
  {\Large \sffamily \textit{\@subtitle}}\par
  }
}
\makeatother

% Keep the header when inserting the title
\newcommand\makefancytitle{%
  \maketitle\vspace{0em plus 2em minus 2em}%
  \thispagestyle{fancy}%
}

% Include graphics in various formats
\usepackage{graphicx}


% Default title; use subtitle for assignment specific titles
\title{Compiler Construction 2019}
\subtitle{Semester project}
\begin{document}

\makefancytitle

\section{Implementing a pretty printer}

For the remainder of the semester, you will be working on an implementation of a Mini Java compiler. In this first part, your task is to familiarise yourself with the template and implement a pretty printer that outputs the source code according to given indentation and formatting rules. The provided JUnit tests are used to verify that your implementation works as expected.

To solve this exercise, do the follwoing tasks.

\begin{itemize} 
\item Implement a front end for the Mini Java grammar at\\
  \url{http://www.cambridge.org/resources/052182060X/MCIIJ2e/grammar.htm}.

\item Implement a pretty printer (\url{http://en.wikipedia.org/wiki/Prettyprint}) for the given Mini Java grammar that conforms to the provided test suite.

\item The provided test suite specifies the pretty printer rules and is mostly for you to get a feel for the number of features you have implemented.
  It will be used by us to do an initial evaluation of your work, but we reserve the right to use additional test cases and other methods for grading.
  You are not allowed to modify the test cases in any way.

\item You can use the following tools.
  \begin{itemize}
  \item Eclipse. Mandatory.
  \item javacc and its Eclipse plugin, for scanner and parser generation. Mandatory.\\
    \url{https://sourceforge.net/projects/eclipse-javacc/}.
  \item JTB, for generation of concrete syntax tree and visitors. Optional.\\
    \url{http://www.cs.ucla.edu/~palsberg/jtb/}
  \end{itemize}

\item You have time until \todo{31 March 2017, 10:00} to complete this part of the project.

\item Push your solution (the Eclipse project, including the parts provided by us) into the your Bitbucket repository. Share the repository with the user \todo{\emph{scg-uko} update user}. 

\item You are not allowed to use other people’s work.
  All parts of the project solution must be your own work.

\item Not all Mini Java features are required for this assignment, but parts of the upcoming project assignments are dependent on the features tested in this assignment, so you are encouraged to do as much as possible. 
\end{itemize}
\end{document}

%%% Local Variables:
%%% mode: latex
%%% TeX-master: t
%%% End: