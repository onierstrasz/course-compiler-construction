\documentclass[a4wide]{scrartcl}


% Page setup
%%%%%%%%%%%%
\usepackage{geometry}
\geometry{
  bottom=2.5cm,
  top=2.5cm,
  left=2.5cm,
  right=2.5cm,
  includeheadfoot
}

%% Fonts (requires lualatex or xelatex!)
%%%%%%%%%%%%%%%%%%%%%%%%%%%%%%%%%%%%%%%%
%\usepackage{fontspec}
%\setmonofont{DejaVu Sans Mono}

% Header and footer
%%%%%%%%%%%%%%%%%%%
\usepackage{fancyhdr}
\lhead{\small Compiler Construction \\ Spring Semester 2019}
\rhead{\small Prof. Dr. O. Nierstrasz, Dr. Mohammad Ghafari\\
        Manuel Leuenberger, Claudio Corrodi}
\cfoot{\thepage}
\usepackage{tabularx}
\usepackage[hidelinks]{hyperref}
\renewcommand{\headrulewidth}{0.1pt}
\renewcommand{\footrulewidth}{0.1pt}
\pagestyle{fancy}


%% Comments, notes, etc.
%%%%%%%%%%%%%%%%%%%%%%%%
\usepackage{ifthen}
\newboolean{showcomments}
\setboolean{showcomments}{true}
%\setboolean{showcomments}{false}

\usepackage{xcolor}
\usepackage{amssymb}

\ifthenelse{\boolean{showcomments}}{
  % then 
  \newcommand{\bnote}[2]{
    \fbox{\bfseries\sffamily\scriptsize#1}
    {\sffamily\small$\blacktriangleright$\textit{#2}$\blacktriangleleft$}
  }
}{
  % else
  \newcommand{\bnote}[2]{}{}
}
  
\newcommand{\triangles}[1]{{\sffamily\small$\blacktriangleright$\textit{#1}$\blacktriangleleft$}}
\newcommand{\nbc}[3]{
  \ifthenelse{\boolean{showcomments}}{
    % then    
    {\colorbox{#3}{\bfseries\sffamily\scriptsize\textcolor{white}{#1}}}%
    {\textcolor{#3}{\sffamily\small$\blacktriangleright$\textit{#2}$\blacktriangleleft$}}
  }{
    % else
  }
}

% Notes
\newcommand\cc[1]{\nbc{CC}{#1}{green!50!black}}
\newcommand\ml[1]{\nbc{ML}{#1}{cyan!80!black}}
\newcommand\todo[1]{\nbc{TODO}{#1}{red}}
\newcommand\verify[1]{\nbc{TO VERIFY}{#1}{red!50!black}}

% Code listings
%%%%%%%%%%%%%%%
\usepackage{listings}
\lstset{
  basicstyle=\ttfamily
}


% Sane quotes
%%%%%%%%%%%%%
\usepackage{csquotes}


% Shorter titles
%%%%%%%%%%%%%%%%
\makeatletter
\def\@maketitle{%
  \noindent
  {\centering
  {\LARGE \sffamily \textbf{\@title}}\\[0.75em]
  {\Large \sffamily \textit{\@subtitle}}\par
  }
}
\makeatother

% Keep the header when inserting the title
\newcommand\makefancytitle{%
  \maketitle\vspace{0em plus 2em minus 2em}%
  \thispagestyle{fancy}%
}

% Include graphics in various formats
\usepackage{graphicx}


% Default title; use subtitle for assignment specific titles
\title{Compiler Construction 2019}
\subtitle{Assignment 1: Introduction}
\begin{document}

\makefancytitle

\section{Important Remarks}

\begin{itemize}
\item Scoring scheme: Assignments -- 10\%, Project -- 30\%, Exam -- 60\%.
\item Assignments: There will be three assignments during the semester.
  \begin{itemize}
  \item \verify{On Friday after the lecture (12:00 - 13:00)} exercises will be posted on the CC website \url{http://scg.unibe.ch/teaching/cc/} and on Piazza. 
  \item Solutions are to be provided by posting a solution into the appropriate folder on Piazza no later than the start of the next lecture \verify{(next Friday at 10:00)}.
  \item Solutions handed in later will get one grade subtracted for each 24h period. In case of serious reasons, send us a message before the deadline.
  \item Assignments and the project will be done \underline{in pairs}.
  \end{itemize}
\item Project: There will be several milestones during the semester.
  \begin{itemize}
  \item There will be a few milestones during the semester. Details will be given when the project requirements are published.
  \item You are expected to create a private git repository on \url{https://bitbucket.org}, to which the answers of the exercises and the code have to be committed.
  \item The solutions of the exercises are to be delivered before \verify{Friday at 10:15}. The commit timestamp will be used to determine if you delivered before the deadline or not. Solutions handed in later will get one grade subtracted for each 24h period (in case of serious reasons send us a message).
  \end{itemize}
\item Exam: The entire course material is covered in the exam. The exam is taken \underline{individually}.
\end{itemize}


\section{How to Qualify for ECTS points}

\begin{itemize}
\item Each series is rated according to the usual rating system with a scale from 1 to 6, with the following meaning:\\
  6 - excellent, 5 - good, 4 - sufficient, 3 - not sufficient, 2 - poor, and 1 - no solution provided.
\item To qualify for the ETCS points, the average mark of all series is required to be a least \emph{sufficient} (i.e., 4). If you are not able to do a series (military service, illness, etc.), let us know as soon as possible.
\item Do not copy solutions from others (nor from the internet). If you really cannot figure out something yourself, discuss it with us or post to the mailing list (see below \todo{no mailing list mentionned?}). In case of copied solutions, you will get a mark of 1.
\end{itemize}

\section{Questions and Discussions:}

\begin{itemize}
\item Use the Piazza platform.
\item Post questions and discussions of general interest to the entire class. 
\item For specific questions and solutions, send a mail with all instructors as recepients. \todo{Insert mail addresses or mailing list.}
\end{itemize}


\clearpage
\section{Assignment 1 - Part I}
\begin{itemize}
\item Register on \url{https://piazza.com/unibe.ch} to class \verify{\textbf{CC 21025}} (select \emph{Spring 2019} as team). \todo{Update link.}
\item You have to use your \texttt{unibe.ch} email address. If you don't have one, send an email to \emph{leuenberger@inf.unibe.ch}, we will add you manually. \todo{check/update mail address}
\item Find a partner for the assignments and the project. Send a message to folder \emph{assignment1} to \verify{CC 21025 instructors} stating name, email address, and student ID number (Matrikelnummer) for both students of the pair.
\end{itemize}


\section{Assignment 1 - Part II}

\begin{itemize}
\item Create a private git repository on \url{https://bitbucket.org} named\\
  \texttt{cc-2019-project-<your-name>-<your-partner-name>}.
\item Share it with user \emph{scg-uko} \todo{update bitbucket user name}.
\item Send a private message containing your and your partner's names, matriculation numbers, and the link for cloning the repository to the appropriate folder (\emph{assignment1}) on Piazza.
\item This exercise is mandatory!
\end{itemize}

\section{Bonus points}

\begin{itemize}
\item A maximum of 5\% additional points can be obtained in the upcoming exercises and projects.
\item Every week, you are encouraged to send an interesting potential exam question (with an answer) regarding the previous lecture \emph{to the instructors (via Piazza)}.

Example:
\begin{itemize}
\item[Q:] What is the difference between a compiler and an interpreter?
\item[A:] A compiler translates a program in one language into a program in another language. An interpreter reads a program and produces the results of running that program.
\end{itemize}

\item The authors of the most interesting questions (as determined by the instructors) will be awarded the additional points.
\item A selection of the questions might appear in the exam.
\end{itemize}

\end{document}

%%% Local Variables:
%%% mode: latex
%%% TeX-master: t
%%% End:
