\documentclass[a4wide]{scrartcl}


% Page setup
%%%%%%%%%%%%
\usepackage{geometry}
\geometry{
  bottom=2.5cm,
  top=2.5cm,
  left=2.5cm,
  right=2.5cm,
  includeheadfoot
}

%% Fonts (requires lualatex or xelatex!)
%%%%%%%%%%%%%%%%%%%%%%%%%%%%%%%%%%%%%%%%
%\usepackage{fontspec}
%\setmonofont{DejaVu Sans Mono}

% Header and footer
%%%%%%%%%%%%%%%%%%%
\usepackage{fancyhdr}
\lhead{\small Compiler Construction \\ Spring Semester 2019}
\rhead{\small Prof. Dr. O. Nierstrasz, Dr. Mohammad Ghafari\\
        Manuel Leuenberger, Claudio Corrodi}
\cfoot{\thepage}
\usepackage{tabularx}
\usepackage[hidelinks]{hyperref}
\renewcommand{\headrulewidth}{0.1pt}
\renewcommand{\footrulewidth}{0.1pt}
\pagestyle{fancy}


%% Comments, notes, etc.
%%%%%%%%%%%%%%%%%%%%%%%%
\usepackage{ifthen}
\newboolean{showcomments}
\setboolean{showcomments}{true}
%\setboolean{showcomments}{false}

\usepackage{xcolor}
\usepackage{amssymb}

\ifthenelse{\boolean{showcomments}}{
  % then 
  \newcommand{\bnote}[2]{
    \fbox{\bfseries\sffamily\scriptsize#1}
    {\sffamily\small$\blacktriangleright$\textit{#2}$\blacktriangleleft$}
  }
}{
  % else
  \newcommand{\bnote}[2]{}{}
}
  
\newcommand{\triangles}[1]{{\sffamily\small$\blacktriangleright$\textit{#1}$\blacktriangleleft$}}
\newcommand{\nbc}[3]{
  \ifthenelse{\boolean{showcomments}}{
    % then    
    {\colorbox{#3}{\bfseries\sffamily\scriptsize\textcolor{white}{#1}}}%
    {\textcolor{#3}{\sffamily\small$\blacktriangleright$\textit{#2}$\blacktriangleleft$}}
  }{
    % else
  }
}

% Notes
\newcommand\cc[1]{\nbc{CC}{#1}{green!50!black}}
\newcommand\ml[1]{\nbc{ML}{#1}{cyan!80!black}}
\newcommand\todo[1]{\nbc{TODO}{#1}{red}}
\newcommand\verify[1]{\nbc{TO VERIFY}{#1}{red!50!black}}

% Code listings
%%%%%%%%%%%%%%%
\usepackage{listings}
\lstset{
  basicstyle=\ttfamily
}


% Sane quotes
%%%%%%%%%%%%%
\usepackage{csquotes}


% Shorter titles
%%%%%%%%%%%%%%%%
\makeatletter
\def\@maketitle{%
  \noindent
  {\centering
  {\LARGE \sffamily \textbf{\@title}}\\[0.75em]
  {\Large \sffamily \textit{\@subtitle}}\par
  }
}
\makeatother

% Keep the header when inserting the title
\newcommand\makefancytitle{%
  \maketitle\vspace{0em plus 2em minus 2em}%
  \thispagestyle{fancy}%
}

% Include graphics in various formats
\usepackage{graphicx}


% Default title; use subtitle for assignment specific titles
\title{Compiler Construction 2019}

\subtitle{Semester project}

\begin{document}

\makefancytitle

\setcounter{section}{2}
\section{Java byte code generation and optimization}

\begin{itemize} 
\item Implement optimization techniques on your intermediate representation of Mini Java. The more, the better. 

\item Implement a code generation unit that transforms your intermediate representation to Java bytecode.
  For the generation of Java byte code use the bcel library (\url{http://commons.apache.org/proper/commons-bcel/}).

  \cc{Hint: Use the \texttt{javap} tool with argument \texttt{-c} (\texttt{javap -c Foo}).}\cc{what?}

\item The provided test suite specifies the code generation rules.
  Use it to ensure that your implementation works as expected and to get a feel for the number of features you have implemented.
  We will use these test cases to evaluate your code generation unit.
  However, we may also use additional tests that check additional scenarios, so you may want to implement your own additional test cases.
  You are not allowed to modify the existing test cases.

\item Import the provided test suite into your eclipse project.

\item You have time until \todo{NN April 2017, 10:00} to complete this part of the project.

\item Push your solution (the Eclipse project, including the parts provided by us) into the your Bitbucket repository. Share the repository with the user \todo{\emph{scg-uko} update user}. 

\item You are not allowed to use other people’s work.
  All parts of the project solution must be your own work.

\end{itemize}


\subsection{Best Bytecode Generator Contest}
Your optimization techniques will be evaluated from the bytecode size point of view.
The best team (teams?) will be awarded with a small reward.

\emph{NB: Make sure you properly use the method \texttt{addMethod(ClassGen cg, MethodGen mg)}, which counts the number of bytecodes.}

\end{document}

%%% Local Variables:
%%% mode: latex
%%% TeX-master: t
%%% End: